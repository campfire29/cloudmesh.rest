\section*{Executive Summary}

The {\it NIST Big Data Interoperability Framework (NBDIF): Volume 8, Reference Architecture Interfaces} document
\cite{nist-vol-8} was prepared by the NIST Big Data Public Working
Group (NBD-PWG) Interface Subgroup to identify interfaces in support
of the NIST Big Data Reference Architecture (NBDRA) The interfaces
contain two different aspects:

\begin{itemize}

\item The definition of resources that are part of the NBDRA. These
  resources are formulated in JSON format and can be integrated into a
  REST framework or an object based framework easily.

\item The definition of simple interface use cases that allow us to
  illustrate the usefulness of the resources defined.

\end{itemize} 

The resources were categorized in groups that are identified by the
NBDRA set forward in the {\it NBDIF: Volume 6, Reference Architecture} document. While 
the {\it NBDIF: Volume 3, Use Cases and Requirements} document provides {\it
  application} oriented high level use cases the use cases defined in
this document are subsets of them and focus on {\it interface} use
cases. The interface use cases are not meant to be complete examples,
but showcase why the resource has been defined. Hence, the interfaces
use cases are, of course, only representative, and do not represent
the entire spectrum of Big Data usage. All of the interfaces were
openly discussed in the working group. Additions are welcome and we
like to discuss your contributions in the group.

The NBDIF consists of nine
volumes, each of which addresses a specific key topic, resulting from
the work of the NBD-PWG. The eight volumes are:

\begin{itemize}
\item Volume 1: Definitions
\item Volume 2: Taxonomies
\item Volume 3: Use Cases and General Requirements
\item Volume 4: Security and Privacy
\item Volume 5: Architectures White Paper Survey
\item Volume 6: Reference Architecture
\item Volume 7: Standards Roadmap
\item Volume 8: Interfaces
\item Volume 9: Big Data Adoption and Modernization
\end{itemize}

The NBDIF will be released in three
versions, which correspond to the three development stages of the
NBD-PWG work. The three stages aim to achieve the following with
respect to the NBDRA.

\begin{quote}
\begin{description}
\item[Stage 1:] Identify the high-level Big Data reference architecture
  key components, which are technology-, infrastructure-, and
  vendor-agnostic.
\item[Stage 2:] Define general interfaces between the NBDRA components.
\item[Stage 3:] Validate the NBDRA by building Big Data general
  applications through the general interfaces.
\end{description}
\end{quote}

This document is targeting Stage 2 of the NBDRA. Coordination of the
group is conducted on its Web page \cite{www-nbdwg}. 
